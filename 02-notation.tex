\section{記法とニューラルネットワークの定義式}
\label{notation}
$\mathcal{D} \subseteq \mathbb{R}^{n_0} \times \mathbb{R}^{k}$をデータセットの集合とし,$\mathcal{X}=\{\bm{x} \ | \ (\bm{x},\bm{y}) \in \mathcal{D}\}$ と $\mathcal{Y}=\{\bm{y} \ | \ (\bm{x},\bm{y}) \in \mathcal{D}\}$をそれぞれ入力データとラベルとする.中間層が$L$層,各層の幅を $n_l  \ (l=1,\ldots,L)$ とし,出力層の幅 (クラス数) を$n_{L+1}=k$とする.入力 $\bm{x} \in \mathbb{R}^{n_0}$ に対し,$h^{l}(\bm{x}), x^{l}(\bm{x}) \in \mathbb{R}^{n_l}$をpre-activation function,post-activation function とする.このとき,ニューラルネットワーク(neural network; NN)の再帰関係の定義式を,
\begin{align}
\label{ntk_parametrization}
    \begin{dcases}
        h^{l+1} = x^{l}\bm{W}^{l+1} + \bm{b}^{l+1} \\
        x^{l+1} = \varphi(h^{l+1})
    \end{dcases}
    \mbox{ and }
    \begin{dcases}
        W_{ij}^{l} = \dfrac{\sigma_{\omega}}{\sqrt{n_l}}\omega_{ij}^{l} \\
        b_{j}^{l} = \sigma_b \beta_{j}^{l}
    \end{dcases}
\end{align}
とする.ここで,$\varphi:\mathbb{R} \rightarrow \mathbb{R}$ はリプシッツ連続 (Lipschitz continuous) かつ2回連続微分可能な要素毎の活性化関数 (element-wise activation function)であり,$\bm{W}^{l+1} \in \mathbb{R}^{n_{l} \times n_{l+1}}$ と $\bm{b}^{l+1} \in \mathbb{R}^{n_{l+1}}$ は重み行列とバイアスベクトルを表し,$\omega_{ij}^{l},\beta_{j}^{l}$ は標準ガウス分布 $\mathcal{N}(0, 1)$ に従って初期化される.$\sigma_{\omega}^2, \sigma_{b}^2$ は重みとバイアスの分散である.この定義は標準的なNNの再帰関係の式
\begin{align}
    \begin{dcases}
        h^{l+1} = x^{l}\bm{W}^{l+1} + \bm{b}^{l+1} \\
        x^{l+1} = \varphi(h^{l+1}) \\
        W_{ij}^{l}, b_{j}^{l} \sim \mathcal{U}\left(-\sqrt{k}, \sqrt{k}\right), \quad k = \dfrac{6}{n_l + n_{l-1}}
    \end{dcases}
\end{align}
とは異なり,NTK parametrizationと呼ばれる(ここでは重みとバイアスの初期化にGlorotの一様分布(\citealp{glorot2010understanding})を用いている).NTK parametrization はNNの順伝播時のダイナミクスのみでなく,誤差逆伝播 (backpropagation) 時のダイナミクスも正規化している.

また,各層$l$毎のパラメータベクトル $\bm{\theta}^l \in \mathbb{R}^{(n_{l-1}+1)n_l}$ とNNの全パラメータベクトル $\bm{\theta} \in \mathbb{R}^P \ (P=\sum_{l=0}^{L-1}(n_{l}+1)n_{l+1})$ を以下のように定義する.
\begin{align}
    \bm{\theta}^{l} \equiv \mbox{vec}\left(\{\bm{W}^{l}, \bm{b}^{l}\}\right), \quad \bm{\theta} = \mbox{vec}\left(\bigcup_{l=1}^L \bm{\theta}^l\right)
\end{align}
ここで$\mbox{vec}(\cdot)$は,行列の各列を縦に並べ,1つの列ベクトルの形にするベクトル化を表す.連続時間 $t \in \mathbb{R}_0^{+}$ ($\mathbb{R}_0^{+}$は0を含む正の実数の集合)  におけるパラメータの時間依存を$\bm{\theta}_t$,その初期値を$\bm{\theta}_0$とし,NNの出力を$f_t(\bm{x}) \equiv h_t^{L+1}(\bm{x}) \in \mathbb{R}^{k}$とする.$\hat{\bm{y}}$をNNによる予測値とすると,損失関数 (loss function) は$\ell (\hat{\bm{y}}, \bm{y}):\mathbb{R}^{k} \times \mathbb{R}^k \rightarrow \mathbb{R}$となる.教師あり学習 (supervised learning) では,以下に記述する経験損失 (empirical loss) $\mathcal{L}$ を最小化する $\bm{\theta}$の学習を行う.
\begin{align}
    \mathcal{L}_t = \dfrac{1}{k|\mathcal{D}|}\sum_{(\bm{x},\bm{y}) \in \mathcal{D}} \ell(f_t(\bm{x},\bm{\theta}_t), \bm{y})
\end{align}