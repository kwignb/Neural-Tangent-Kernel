\section{中間層が無限幅のネットワーク}
\label{infinite_width}
3節ではNTKの定義を与えたが,一般には関数空間内での学習を考える式(\ref{func_learning})の計算は難しい.何故なら,NTK $\hat{\Theta}_t$ は時間毎に変化するためである.しかし,損失関数を平均二乗誤差 (mean square error; MSE),$\lambda_{\min/\max}(\Theta)$ をNTK $\Theta$ の最小/最大固有値,$\eta_{\text{critical}} \coloneqq 2(\lambda_{\min}(\Theta)+\lambda_{\max}(\Theta))^{-1}$ としたとき,以下の定理が証明されている.
\begin{thm}[(\citealp{lee2019wide})]
\label{statement_ntk}
    $n_1 = \cdots = n_L = n$とし,$\lambda_{\min}(\Theta) > 0$を仮定する.学習率 $\eta$ が $\eta_{\text{critical}}$ より小さく,任意の入力 $\bm{x} \in \mathbb{R}^{n_0}$ が $\|\bm{x}\|_2 \leq 1$ を満たすとき,以下が成立する.
    \begin{align*}
        \sup_{t \geq 0} \|f_t(\bm{x}) - f_t^{lin}(\bm{x})\|_2, \ \sup_{t \geq 0} \dfrac{\|\bm{\theta}_t - \bm{\theta}_0\|_2}{\sqrt{n}}, \ \sup_{t \geq 0}\left\|\hat{\Theta}_t - \hat{\Theta}_0\right\|_F = \mathcal{O}\left(n^{-\frac{1}{2}}\right), \ \mbox{as} \quad n \rightarrow \infty
    \end{align*}
\end{thm}
\begin{proof}

\end{proof}
\par
\noindent
$f_t^{lin}(\bm{x})$ について,詳細は式(\ref{f_t_lin})で示すがNTK $\hat{\Theta}_0$ を用いて記述されたモデルの出力である.この定理は中間層の幅を無限にしたとき,$t$ を大きく取ったときにも 学習の最適なパラメータ $\bm{\theta}_t$ は $\bm{\theta}_0$ の近傍にあることを主張しており,さらに $\Theta_t$ を $\Theta_0$ で近似できることも主張している.この定理を用いて,NTKを用いたネットワークの出力ダイナミクスの記述を行う.