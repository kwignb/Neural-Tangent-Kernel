\section{NTKを用いたネットワークの出力ダイナミクス}
\label{NTK_dynamics}
本節ではニューラルネットワークの出力を1次のテイラー展開 (Taylor expansion) で置き換えることにより,線形化されたニューラルネットワーク (linearized neural network) の学習のダイナミクスを記述する.時間 $t$ 時点における線形化されたネットワークの出力を $f_t^{lin}(\bm{x})$ とすると,1次のテイラー展開は,
\begin{align}
    f_t^{lin}(\bm{x}) \equiv f_0(\bm{x}) + \nabla_{\bm{\theta}}f_0(\bm{x})|_{\bm{\theta}=\bm{\theta}_0}\bm{\omega}_t \label{f_t_lin}
\end{align}
と書ける.ここで,$\bm{\omega}_t=\bm{\theta}_t - \bm{\theta}_0$である.$f_t^{lin}$ は第一項がネットワークの初期値であり,第二項が学習中の初期値からの値の変化を示す.$\bm{\omega}_t$ の時間微分は
\begin{align}
    \dot{\bm{\omega}}_t = -\eta \nabla_{\bm{\theta}}f_0(\mathcal{X})^{\mathrm{T}}\nabla_{f_t^{lin}(\mathcal{X})}\mathcal{L} \label{ode}
\end{align}
となる.$\nabla_{\bm{\theta}}f_0(\bm{x})$は学習を通して一定の値を取るので,損失関数にMSE,すなわち$\ell(\hat{\bm{y}},\bm{y}) = \frac{1}{2}|\hat{\bm{y}}-\bm{y}|_2^2$を使用したとき,式(\ref{ode})について常微分方程式を解けば,
\begin{align}
    \bm{\omega}_t = -\nabla_{\bm{\theta}}f_0(\mathcal{X})^{\mathrm{T}}\hat{\Theta}_{0}^{-1}\left(I - e^{-\eta\hat{\Theta}_0 t/k|\mathcal{D}|}\right)(f_0(\mathcal{X})- \mathcal{Y})
\end{align}
が得られる.つまり,$f_t^{lin}(\mathcal{X})$ は,
\begin{align}
    f_t^{lin}(\mathcal{X}) &= f_0(\mathcal{X}) - \nabla_{\bm{\theta}}f_0(\mathcal{X})\nabla_{\bm{\theta}}f_0(\mathcal{X})^{\mathrm{T}}\hat{\Theta}_{0}^{-1}\left(I - e^{-\eta\hat{\Theta}_0 t/k|\mathcal{D}|}\right)(f_0(\mathcal{X})- \mathcal{Y}) \\
    &= f_0(\mathcal{X}) - \hat{\Theta}_0 \hat{\Theta}_0^{-1}\left(I - e^{-\eta\hat{\Theta}_0 t/k|\mathcal{D}|}\right)(f_0(\mathcal{X})- \mathcal{Y}) \\
    &= f_0(\mathcal{X}) - \left(I - e^{-\eta\hat{\Theta}_0 t/k|\mathcal{D}|}\right)(f_0(\mathcal{X})- \mathcal{Y}) \\
    &= \left(I - e^{-\eta\hat{\Theta}_0 t/k|\mathcal{D}|}\right)\mathcal{Y} + e^{-\eta\hat{\Theta}_0 t/k|\mathcal{D}|}f_0(\mathcal{X}) \label{ntk_output}
\end{align}
と書ける.$\mathcal{X}$ 以外の入力 $\bm{x}'$ に対する出力 $f_t^{lin}(\bm{x}')$ は,
\begin{align}
    f_t^{lin}(\bm{x}') &= f_0(\bm{x}') - \nabla_{\bm{\theta}}f_0(\bm{x}')\nabla_{\bm{\theta}}f_0(\mathcal{X})^{\mathrm{T}}\hat{\Theta}_{0}^{-1}\left(I - e^{-\eta\hat{\Theta}_0 t/k|\mathcal{D}|}\right)(f_0(\mathcal{X})- \mathcal{Y}) \\
    &= f_0(\bm{x}') - \hat{\Theta}_0(\bm{x}', \mathcal{X}) \hat{\Theta}_0^{-1}\left(I - e^{-\eta\hat{\Theta}_0 t/k|\mathcal{D}|}\right)(f_0(\mathcal{X})- \mathcal{Y})
\end{align}
と書ける.したがって,初期化時のネットワークの出力 $f_0(\mathcal{X}), \ f_0(\bm{x}')$ とNTK $\hat{\Theta}_0, \ \hat{\Theta}_0(\bm{x}',\mathcal{X})$を計算すれば, 勾配降下法を実行することなく,線形化されたニューラルネットワークの各時間における学習結果を計算できるということである.通常であれば,このようにネットワークの出力を構成したとしても,各時間における NTK $\hat{\Theta}_t$ を計算する必要があり,何もメリットがない.しかし,4.3節で述べた定理\ref{statement_ntk}により,中間層の幅を十分に大きく取れば,任意の学習時間における NTK $\hat{\Theta}_t$ を $\hat{\Theta}_0$ で近似することができるため,関数空間内においてネットワークを線形化することができる.